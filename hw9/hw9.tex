\documentclass[12pt,AutoFakeBold]{article}
\usepackage[a4paper]{geometry}%纸张大小和页边距
\geometry{left=2.0cm,right=2.0cm,top=2.0cm,bottom=2.0cm}%页边距设置

\title{AI引论第九次大班课作业}
\author{信息科学技术学院\ 施朱鸣\ 1800011723}
\date{2020年4月11日}

\usepackage{setspace}% 控制行距
\renewcommand{\baselinestretch}{1.5}% 1.5倍行距

\usepackage{fontspec}%控制字体
\setmainfont{Times New Roman}%英文字体
% \newfontfamily\arial{Arial}%Arial字体
\usepackage{xeCJK}%控制中文
\setCJKmainfont{SimSun}%中文字体
\XeTeXlinebreaklocale "zh"%中文自动换行
\XeTeXlinebreakskip = 0pt plus 1pt%中文自动换行
\usepackage{indentfirst}%中文缩进
\setlength{\parindent}{2em}

\usepackage{amsthm,amsmath,amssymb,mathrsfs}%数学符号和花体支持
\usepackage{booktabs}%绘制三线表
\usepackage{latexsym}%绘制特殊数学符号
\usepackage{siunitx}%数学模式中使用SI单位

\usepackage{xcolor}%高亮使用的颜色
\definecolor{commentcolor}{RGB}{85,139,78}
\definecolor{stringcolor}{RGB}{206,145,108}
\definecolor{keywordcolor}{RGB}{34,34,250}
\definecolor{backcolor}{RGB}{220,220,220}

\usepackage{accsupp}%不知道是什么
\newcommand{\emptyaccsupp}[1]{\BeginAccSupp{ActualText={}}#1\EndAccSupp{}}

\usepackage{listings}%代码高亮
\lstset{
    language=python, 					%Python语法高亮
	linewidth=0.9\linewidth,      		%列表list宽度
	%basicstyle=\ttfamily,				%tt无法显示空格
	commentstyle=\color{commentcolor},	%注释颜色
	keywordstyle=\color{keywordcolor},	%关键词颜色
	stringstyle=\color{stringcolor},	%字符串颜色
	%showspaces=true,					%显示空格
	numbers=left,						%行数显示在左侧
	numberstyle=\tiny\emptyaccsupp,		%行数数字格式
	numbersep=5pt,						%数字间隔
	frame=single,						%加框
	framerule=0pt,						%不划线
	escapeinside=@@,					%逃逸标志
	emptylines=1,						%
	xleftmargin=3em,					%list左边距
	backgroundcolor=\color{backcolor},	%列表背景色
	tabsize=4,							%制表符长度为4个字符
	gobble=4							%忽略每行代码前4个字符                                        % 设置语言
}

\begin{document}
    \maketitle
    \section*{第1题}
    进行1个单位宽度的0-边界填充后的图像如下
    \begin{center}
        \begin{tabular}{|c|c|c|c|c|c|}
            % \centering
            \hline
            0&0&0&0&0&0\\
            \hline
            0&0&1&1&0&0\\
            \hline
            0&0&1&1&0&0\\
            \hline
            0&0&2&2&0&0\\
            \hline
            0&1&1&1&1&0\\
            \hline
            0&0&0&0&0&0\\
            \hline
        \end{tabular}
    \end{center}

    以1个单元为步长,对每个点做卷积
    \[y=\sum_i x_i\textbf{w}_i\]

    得到如下结果,即为\(I'\)各个位置对应的像素值
    \begin{center}
        \begin{tabular}{|c|c|c|c|}
            \hline
            -2&-2&2&2\\
            \hline
            -3&-3&3&3\\
            \hline
            -3&-2&2&3\\
            \hline
            -1&0&0&1\\
            \hline
        \end{tabular}
    \end{center}

    \section*{第2题}
    对于每个像素点,其\(\hat{I}-I'\)对应的矩阵如下

    \begin{center}
        \begin{tabular}{|c|c|c|c|}
            \hline
            0&0&0&0\\
            \hline
            -1&-1&1&1\\
            \hline
            -1&-1&1&1\\
            \hline
            -2&-2&2&2\\
            \hline
        \end{tabular}
    \end{center}

    用\(\mathcal{L}\)表示均方差损失函数,则
    \begin{equation}
        \begin{split}
            \mathcal{L} &= \frac{1}{16}\sum_{(m,n)} (\hat{x}_{m,n} - {x'}_{m,n})^2\\
            &= \frac{1}{16}\left(0^2 + 0^2 + 0^2 + 0^2 + 1^2 + 1^2 +1^2 +1^2 +1^2 + 1^2 +1^2 +1^2 + 2^2 + 2^2 + 2^2 + 2^2 \right)\\
            &= \frac{3}{2}\\
            &= 1.5
        \end{split}
    \end{equation}

    假设\(K_{i,j}\)中的\(i,j\)都从0开始取值,则

    \begin{equation}
        \mathcal{L} = \frac{1}{16}\sum_{1 \leq m,n \leq 4} (\hat{x}_{m,n} - {x'}_{m,n})^2
    \end{equation}

    \begin{equation}
        \begin{split}
            \frac{\partial \mathcal{L}}{\partial K_{i,j}} 
            &= \frac{1}{16}\sum_{1\leq m,n \leq 4} \frac{\partial \left(\hat{x}_{m,n} - {x'}_{m,n}\right)^2}{\partial K_{i,j}}\\
            &= \frac{1}{16}\times 2\sum_{1\leq m,n \leq 4} \frac{\partial \left(\hat{x}_{m,n} - {x'}_{m,n}\right)^2}{\partial \left(\hat{x}_{m,n} - {x'}_{m,n}\right)}\frac{\partial \left(\hat{x}_{m,n} - {x'}_{m,n}\right)}{\partial K_{i,j}}\\
            &= \frac{1}{8}\sum_{1\leq m,n \leq 4} \left(\hat{x}_{m,n} - {x'}_{m,n}\right)\times (-1)\frac{\partial {x'}_{m,n}}{\partial K_{i,j}}\\
            &= \frac{1}{8}\sum_{1\leq m,n \leq 4} \left(\hat{x}_{m,n} - {x'}_{m,n}\right)\times (-1)\frac{\partial \sum_{0 \leq p,q\leq 2} x_{m-1+p,m-1+q}K_{p,q}}{\partial K_{i,j}}\\
            &= \frac{1}{8}\sum_{1\leq m,n \leq 4} \left(\hat{x}_{m,n} - {x'}_{m,n}\right)\times (-1)\times x_{m-1+i,m-1+j}
        \end{split}
    \end{equation}

    用上面的计算方法得出卷积核各点的\(\frac{\partial \mathcal{L}}{\partial K_{i,j}}\)如下矩阵

    \begin{center}
        \begin{tabular}{|c|c|c|}
            \hline
            -1.5&0.0&1.5\\
            \hline
            -1.0&0.0&1.0\\
            \hline
            -0.625&0.0&0.625\\
            \hline
        \end{tabular}
    \end{center}

    \section*{第3题}
    对于已经求得的梯度矩阵,利用如下式子更新卷积核参数
    \[\hat{K_{i,j}} = K_{i,j} - \eta\frac{\partial \mathcal{L}}{\partial K_{i,j}}\]

    当\(\eta = 0.01\)时更新后的卷积核参数矩阵如下
    \begin{center}
        \begin{tabular}{|c|c|c|}
            \hline
            0.015&0&-0.015\\
            \hline
            1.01&0&-1.01\\
            \hline
            1.00625&0&-1.00625\\
            \hline
        \end{tabular}
    \end{center}

    再次卷积得到的输出结果如下矩阵
    \begin{center}
        \begin{tabular}{|c|c|c|c|}
            \hline
            -2.01625&-2.01625&2.01625&2.01625\\
            \hline
            -3.0375&-3.0375&3.0375&3.0375\\
            \hline
            -3.04125&-2.035&2.035&3.04125\\
            \hline
            -1.04&-0.03&0.03&1.04\\
            \hline
        \end{tabular}
    \end{center}

    用\(\hat{\mathcal{L}}\)表示更新后的均方差损失函数,则
    \begin{equation}
        \begin{split}
            \hat{\mathcal{L}} &= \sum_{(m,n)} (\hat{x}_{m,n} - {x'}_{m,n})^2\\
            &= 1.428
        \end{split}
    \end{equation}

    \section*{附:实现以上计算的Python代码}
    \begin{lstlisting}[language=python]
    # To add a new cell, type '# %%'
    # To add a new markdown cell, type '# %% [markdown]'
    # %%
    kernel = [
                [0, 0, 0],
                [1, 0, -1],
                [1, 0, -1]]


    # %%
    I = [
            [0, 0, 0, 0, 0, 0],
            [0, 0, 1, 1, 0, 0],
            [0, 0, 1, 1, 0, 0],
            [0, 0, 2, 2, 0, 0],
            [0, 1, 1, 1, 1, 0],
            [0, 0, 0, 0, 0, 0]]


    # %%
    targetI = [
                [-2, -2, 2, 2],
                [-4, -4, 4, 4],
                [-4, -3, 3, 4],
                [-3, -2, 2, 3]]


    # %%
    outputI = [
                [0, 0, 0, 0],
                [0, 0, 0, 0],
                [0, 0, 0, 0],
                [0, 0, 0, 0]]


    # %%
    grad = [
            [0, 0, 0],
            [0, 0, 0],
            [0, 0, 0]]


    # %%
    # 卷积
    for i in range(1, 5):
        for j in range(1, 5):
            outputI[i-1][j-1] = 0
            for k in range(3):
                for l in range(3):
                    outputI[i-1][j-1] += I[i-1+k][j-1+l] * kernel[k][l]
    print(outputI)


    # %%
    # 计算均方差
    MSE = 0
    for i in range(4):
        for j in range(4):
            MSE += (targetI[i][j] - outputI[i][j])**2
    MSE = float(MSE)/16.0
    print(MSE)


    # %%
    for i in range(3):
        for j in range(3):
            grad[i][j] = 0
            for k in range(4):
                for l in range(4):
                    grad[i][j] += ((targetI[k][l] - outputI[k][l])*(-1)*(I[k + i][l + j]))/8.0
    print(grad)


    # %%
    eta = 0.01
    for i in range(3):
        for j in range(3):
            kernel[i][j] -= eta * grad[i][j]
    print(kernel)


    # %%
    for i in range(1, 5):
        for j in range(1, 5):
            outputI[i-1][j-1] = 0
            for k in range(3):
                for l in range(3):
                    outputI[i-1][j-1] += I[i-1+k][j-1+l] * kernel[k][l]
    print(outputI)


    # %%
    MSE = 0
    for i in range(4):
        for j in range(4):
            MSE += (targetI[i][j] - outputI[i][j])**2
    MSE = float(MSE)/16.0
    print(MSE)
    \end{lstlisting}

\end{document}